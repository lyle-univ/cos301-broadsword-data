\documentclass{article}
\usepackage{hyperref}
\usepackage{graphicx}
\hypersetup{
    colorlinks,
    citecolor=black,
    filecolor=black,
    linkcolor=black,
    urlcolor=black
}

\begin{document}
\begin{titlepage}
	{\scshape\LARGE NavUP System\par}
	\vfill
	{\scshape\Large Testing phase\par}
	\vfill
	{\Large Group name: Broadsword Data\par}
	\bigskip
	{\itshape\Large Jacques Smulders - u15003087\par}
	{\itshape\Large Keegan Ferret - u15132847\par}
	{\itshape\Large Lesego Makaleng - u15175716\par}
	{\itshape\Large Linda Potgieter - u14070091\par}
  {\itshape\Large Lyle Nel - u29562695\par}
	{\itshape\Large Seonin David - u15063021\par}
	\vfill
\end{titlepage}

\tableofcontents

\clearpage

\section{Introduction}
This document specifies the steps taken to test the Gladius Implementation of the data subsystem. The data subsystem was required to return a devices location based on its MAC address. The location returned would be the x and y coordinates of the connected device (based on its MAC address) relevant to the Wi-Fi hotspot it is connected to at the time of request. Testing consisted of functional and non-functional requirement tests, with each test case having a description, severity level, a mark obtained out of 10 and a status which may either be Pass or Fail.

\newcounter{FuncReqSerial}

\newcommand{\FuncReq} [5]{
    \refstepcounter{FuncReqSerial}
	\noindent
    \textbf{Serial}:F\theFuncReqSerial\\
    \textbf{Title}: #1\\
    \textbf{Description}: #2\\
    \textbf{Severity}: #3\\
    \textbf{Score}: #4\\
    \textbf{Status}: #5\\
}

\newcounter{NonFuncReqSerial}

\newcommand{\NonFuncReq} [5]{
\refstepcounter{NonFuncReqSerial}
\noindent
\textbf{Serial}:NF\theNonFuncReqSerial\\
\textbf{Title}: #1\\
\textbf{Description}: #2\\
\textbf{Severity}: #3\\
\textbf{Score}: #4\\
\textbf{Status}: #5\\
}

\section{Introduction}
This document specifies the steps taken to test the Gladius Implementation of the data subsystem. The data subsystem was required to return a device's location based on its MAC address. The location returned would be the x and y coordinates of the connected device (based on its MAC address) relevant to the Wi-Fi hotspot it is connected to at the time of request. Testing consisted of functional and non-functional requirement tests, with each test case having a description, severity level, a mark obtained out of 10 and a status which may either be Pass or Fail.


\section{Functional Tests}
\section{Introduction}
This document specifies the steps taken to test the Gladius Implementation of the data subsystem. The data subsystem was required to return a device's location based on its MAC address. The location returned would be the x and y coordinates of the connected device (based on its MAC address) relevant to the Wi-Fi hotspot it is connected to at the time of request. Testing consisted of functional and non-functional requirement tests, with each test case having a description, severity level, a mark obtained out of 10 and a status which may either be Pass or Fail.


\subsection{Tests}
\FuncReq
{Running Apache Flink with defined data sink}
{For this test I ran Apache Flink code that I got from the Gladios Data team's repository.The system's defined data sink is \textbf{\textit{  text.map(new Mapper()).writeAsText("C:/Users/rob/Documents/NetBeansProjects/COS301/test.txt");}} but this line will only work on one computer and did not run when we tested it. As can be seen in \textbf{Figure 1}, the reason why the file directory cannot be found is because it is only specified for one computer.
			\begin{figure}[h]
				\centering
				\includegraphics{wrong_file_error.jpg}
				\caption{Wrong File Path}
				\label{fig:WrongFIlePath}
			\end{figure} }
{Critical}
{0}
{Fail}
		

{Running Flink after changing data sink}
{For this test I ran Apache Flink code but I changed the data sink to \textbf{\textit{  text.map(new Mapper()).writeAsText(/"tmp/test.txt");}}, so that the system can run on any computer. After changing the data source the system runs without any error and returns a devices location in the text file, as can be seen in \textbf{Figure 2}.As a side note is shows that the result is the Longitude and Latitude but it is actually the x and y values relative to the access point.
		\begin{figure}[h]
			\centering
			\includegraphics{query_location_results.jpg}
			\caption{Output}
			\label{fig:Output}
		\end{figure} 
}
		
{N/A}
{8}
{Pass}
		
{Running flink without starting the Job Manager}
{This test will see if Apache Flink runs and returns the relevant information without starting the Job Manager. Flink does run without a Job Manager and returns the same result which can be seen in Figure 2 of the previous test.}
{N/A}
{10}
{Pass}



\section{Functional Tests}
\section{Introduction}
This document specifies the steps taken to test the Gladius Implementation of the data subsystem. The data subsystem was required to return a device's location based on its MAC address. The location returned would be the x and y coordinates of the connected device (based on its MAC address) relevant to the Wi-Fi hotspot it is connected to at the time of request. Testing consisted of functional and non-functional requirement tests, with each test case having a description, severity level, a mark obtained out of 10 and a status which may either be Pass or Fail.


\subsection{Tests}
\FuncReq
{Running Apache Flink with defined data sink}
{For this test I ran Apache Flink code that I got from the Gladios Data team's repository.The system's defined data sink is \textbf{\textit{  text.map(new Mapper()).writeAsText("C:/Users/rob/Documents/NetBeansProjects/COS301/test.txt");}} but this line will only work on one computer and did not run when we tested it. As can be seen in \textbf{Figure 1}, the reason why the file directory cannot be found is because it is only specified for one computer.
			\begin{figure}[h]
				\centering
				\includegraphics{wrong_file_error.jpg}
				\caption{Wrong File Path}
				\label{fig:WrongFIlePath}
			\end{figure} }
{Critical}
{0}
{Fail}
		

{Running Flink after changing data sink}
{For this test I ran Apache Flink code but I changed the data sink to \textbf{\textit{  text.map(new Mapper()).writeAsText(/"tmp/test.txt");}}, so that the system can run on any computer. After changing the data source the system runs without any error and returns a devices location in the text file, as can be seen in \textbf{Figure 2}.As a side note is shows that the result is the Longitude and Latitude but it is actually the x and y values relative to the access point.
		\begin{figure}[h]
			\centering
			\includegraphics{query_location_results.jpg}
			\caption{Output}
			\label{fig:Output}
		\end{figure} 
}
		
{N/A}
{8}
{Pass}
		
{Running flink without starting the Job Manager}
{This test will see if Apache Flink runs and returns the relevant information without starting the Job Manager. Flink does run without a Job Manager and returns the same result which can be seen in Figure 2 of the previous test.}
{N/A}
{10}
{Pass}




\end{document}
