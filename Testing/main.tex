\documentclass{article}
\usepackage{hyperref}
\usepackage{graphicx}
\hypersetup{
    colorlinks,
    citecolor=black,
    filecolor=black,
    linkcolor=black,
    urlcolor=black
}

\begin{document}
\begin{titlepage}
	{\scshape\LARGE NavUP System\par}
	\vfill
	{\scshape\Large Testing phase\par}
	\vfill
	{\Large Group name: Broadsword Data\par}
	\bigskip
	{\itshape\Large Jacques Smulders - u15003087\par}
	{\itshape\Large Keegan Ferret - u15132847\par}
	{\itshape\Large Lesego Makaleng - u15175716\par}
	{\itshape\Large Linda Potgieter - u14070091\par}
  {\itshape\Large Lyle Nel - u29562695\par}
	{\itshape\Large Seonin David - u15063021\par}
	\vfill
\end{titlepage}

\tableofcontents

\clearpage

\section{Introduction}
This document specifies the steps taken to test the Gladius Implementation of the data subsystem. The data subsystem was required to return a devices location based on its MAC address. The location returned would be the x and y coordinates of the connected device (based on its MAC address) relevant to the Wi-Fi hotspot it is connected to at the time of request. Testing consisted of functional and non-functional requirement tests, with each test case having a description, severity level, a mark obtained out of 10 and a status which may either be Pass or Fail.

\newcounter{FuncReqSerial}

\newcommand{\FuncReq} [5]{
    \refstepcounter{FuncReqSerial}
	\noindent
    \textbf{Serial}:F\theFuncReqSerial\\
    \textbf{Title}: #1\\
    \textbf{Description}: #2\\
    \textbf{Severity}: #3\\
    \textbf{Score}: #4\\
    \textbf{Status}: #5\\
}

\newcounter{NonFuncReqSerial}

\newcommand{\NonFuncReq} [5]{
\refstepcounter{NonFuncReqSerial}
\noindent
\textbf{Serial}:NF\theNonFuncReqSerial\\
\textbf{Title}: #1\\
\textbf{Description}: #2\\
\textbf{Severity}: #3\\
\textbf{Score}: #4\\
\textbf{Status}: #5\\
}

\subsection{Introduction}
The functional requirements specify the behaviour of the system. It consists of the abilities that the system must have to be considered functional. The main purpose of the subsystem, whether it will be able to return the location (relevant to a connected Wi-Fi hotspot) of a device with a given MAC address, will be tested. Testing will also include error checking and handling for various components of the subsystem including the Aruba Location Engine, Apache Flink and Json queries.  

\documentclass{article}
\usepackage{hyperref}
\usepackage{graphicx}
\hypersetup{
    colorlinks,
    citecolor=black,
    filecolor=black,
    linkcolor=black,
    urlcolor=black
}
\begin{document}
This is just a moxk main file to verify that the latex markup compiles and that diagrams fit properly.
	\section{Data streaming module}
\subsection{Scope and responsibilities}
The data streaming module serves as an intermediary between the Aruba analytics engine and the rest of the subsystem. The module is mainly concerned with serving requests about the location of a device visible to Aruba.

\subsection{End User}
This section is intended for system administrators and anyone that needs to deploy this subsystem.
\subsubsection{Installation and dependencies}
To satisfy dependencies run
\begin{verbatim}
  make dependency
\end{verbatim}
To install the system, run
\begin{verbatim}
  make install
\end{verbatim}
\subsubsection{Configuration}
The ALE server that exposes the API for the Aruba engine, as well as the NSQ server that acts as a messaging system, will have a host address and port to connect to. These details are placed in the data_subsystem.cfg file or as program arguments depending on what data decides on.
\subsubsection{Troubleshooting}
Any malformed queries made to the data subsystem are logged in a file called error.log for diagnostic purposes. In addition, any connection issues to either the ALE or NSQ server is also be logged.

\subsection{Technical}
\subsubsection{API}
\subsubsection{Sub-architecture}
The push pull relationship between interfaces of different components are given below.
\begin{figure}[h]
    \makebox[\textwidth][c]{\includegraphics[width=1\textwidth]{component_diagram.pdf}}
\caption{Overview of components}
\subsubsection{Implementation details}
\subsubsection{Class diagram}
\begin{figure}[h]
    \makebox[\textwidth][c]{\includegraphics[width=1\textwidth]{class_diagram.pdf}}
\caption{How it fits together}



\end{document}

\documentclass{article}
\usepackage{hyperref}
\usepackage{graphicx}
\hypersetup{
    colorlinks,
    citecolor=black,
    filecolor=black,
    linkcolor=black,
    urlcolor=black
}
\begin{document}
This is just a moxk main file to verify that the latex markup compiles and that diagrams fit properly.
	\section{Data streaming module}
\subsection{Scope and responsibilities}
The data streaming module serves as an intermediary between the Aruba analytics engine and the rest of the subsystem. The module is mainly concerned with serving requests about the location of a device visible to Aruba.

\subsection{End User}
This section is intended for system administrators and anyone that needs to deploy this subsystem.
\subsubsection{Installation and dependencies}
To satisfy dependencies run
\begin{verbatim}
  make dependency
\end{verbatim}
To install the system, run
\begin{verbatim}
  make install
\end{verbatim}
\subsubsection{Configuration}
The ALE server that exposes the API for the Aruba engine, as well as the NSQ server that acts as a messaging system, will have a host address and port to connect to. These details are placed in the data_subsystem.cfg file or as program arguments depending on what data decides on.
\subsubsection{Troubleshooting}
Any malformed queries made to the data subsystem are logged in a file called error.log for diagnostic purposes. In addition, any connection issues to either the ALE or NSQ server is also be logged.

\subsection{Technical}
\subsubsection{API}
\subsubsection{Sub-architecture}
The push pull relationship between interfaces of different components are given below.
\begin{figure}[h]
    \makebox[\textwidth][c]{\includegraphics[width=1\textwidth]{component_diagram.pdf}}
\caption{Overview of components}
\subsubsection{Implementation details}
\subsubsection{Class diagram}
\begin{figure}[h]
    \makebox[\textwidth][c]{\includegraphics[width=1\textwidth]{class_diagram.pdf}}
\caption{How it fits together}



\end{document}


\end{document}
